\defaultfontfeatures{Ligatures=TeX}

\setsansfont{Calibri}
\setmainfont{Calibri}
%设置字体

\graphicspath{{fig/}} %设置图片路径

%\setlength{\parindent}{0pt} %首行无空格

\linespread{1.5} %行间距设置

%\titlecontents{section}[0.8cm]{\bfseries}{\contentslabel{2.5 em}}{}{\titlerule*[0.6pc]{$\cdot$}\contentspage\hspace*{0.8cm}}
%\titlecontents{subsection}[0.8cm]{}{\contentslabel{2.5 em}}{}{\titlerule*[0.6pc]{$\cdot$}\contentspage\hspace*{0.8cm}}
%\titlecontents{subsubsection}[0.8cm]{}{\contentslabel{2.5 em}}{}{\titlerule*[0.6pc]{$\cdot$}\contentspage\hspace*{0.8cm}}
% 设置contents格式

%\titleformat*{\section}{\Large \bfseries}
%\titleformat*{\subsection}{\large \bfseries}
%\titleformat*{\subsubsection}{\normalsize \bfseries}
% 设置section格式

\renewcommand{\abstractnamefont}{\setmainfont{Times New Roman} \bfseries \slshape} % 设置abstract标题格式

\pagestyle{fancy}
\fancyhf{}

%\fancyhead[R]{MTH227 - Final Project} % 页眉内容
% \fancyhead[L]{\ifthenelse{\value{page} = 1}{MTH227 - Final Project}{Squid-like Bionic Robot}} %不同情况页眉不同
% \fancyhead[R]{\date{October 10, 2020}}
\fancyfoot[c]{\thepage\ \textbackslash \ \pageref{LastPage}} % 页脚内容

\renewcommand{\headrulewidth}{0mm} % 页眉线宽
%\renewcommand{\footrulewidth}{0.25mm} % 页脚线宽

\title{A Squid-like Bionic Robot \\ {\footnotesize Note: Sub-titles are not captured in Xplore andshould not be used}}
%\title{A Squid-like Bionic Robot} % 标题
\author{San Zhan} % 作者
%\date{\today}
\date{} % 日期

\lstset{
    alsolanguage = Java,
	%alsolanguage = [ANSI]C,      %可以添加很多个alsolanguage,如alsolanguage=matlab,alsolanguage=VHDL等
	alsolanguage = matlab,
    alsolanguage = Python,
	basicstyle = \small\setmainfont{JetBrains Mono}, %设置字体族
	breaklines = true, %自动换行
	keywordstyle = \bfseries\color{NavyBlue}, %设置关键字为粗体,颜色为 NavyBlue
	morekeywords = {}, %设置更多的关键字,用逗号分隔
	emph = {self}, %指定强调词,如果有多个,用逗号隔开
    emphstyle = \bfseries\color{Rhodamine}, %强调词样式设置
    commentstyle = \color{black!50!white}, %设置注释样式,浅灰色
    stringstyle = \bfseries\color{PineGreen!90!black}, %设置字符串样式
    columns = flexible, %使代码更紧凑
    %numbers = left, %显示行号在左边
    %numbersep = 2em, %设置行号的具体位置
    %numberstyle = \footnotesize, % 缩小行号
    frame = single, %边框
    framesep = 1 em, %设置代码与边框的距离
}

\hypersetup{
    linktocpage = false,
    colorlinks = true,
    linkcolor = blue,
    filecolor = magenta,
    citecolor = red,
    urlcolor = 	cyan,
    anchorcolor = black
}

\urlstyle{same}