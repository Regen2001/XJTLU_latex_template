\defaultfontfeatures{Ligatures=TeX}

%\setsansfont{Tahoma}
%\setmainfont{Tahoma}
%设置字体

\graphicspath{{fig/}} %设置图片路径

%\setlength{\parindent}{0pt} %首行无空格

%\linespread{1.5} %行间距设置

%\setlength\columnsep{0.5cm} %栏间距设置

%\titlecontents{section}[0.8cm]{\bfseries}{\contentslabel{2.5 em}}{}{\titlerule*[0.6pc]{$\cdot$}\contentspage\hspace*{0.8cm}}
%\titlecontents{subsection}[0.8cm]{}{\contentslabel{2.5 em}}{}{\titlerule*[0.6pc]{$\cdot$}\contentspage\hspace*{0.8cm}}
%\titlecontents{subsubsection}[0.8cm]{}{\contentslabel{2.5 em}}{}{\titlerule*[0.6pc]{$\cdot$}\contentspage\hspace*{0.8cm}}
% 设置contents格式

%\titleformat*{\section}{\centering \bfseries}
%\titleformat*{\subsection}{\bfseries}
%\titleformat*{\subsubsection}{\normalsize}
% 设置section格式

%\renewcommand{\abstractnamefont}{\setmainfont{Times New Roman} \bfseries \slshape} % 设置abstract标题格式

\pagestyle{fancy}
\fancyhf{}

%\fancyhead[L]{MEC303 - Assignment 1} % 页眉内容
%\fancyhead[L]{\ifthenelse{\value{page} = 1}{MEC303 - Assignment 1}{Robot Development and Spatial Transformations}} %不同情况页眉不同
%\fancyhead[R]{\ifthenelse{\value{page} = 1}{Xi'an Jiaotong-liverpool University}{School of advanced technology}} %不同情况页眉不同
\fancyfoot[C]{\thepage\ \textbackslash \ \pageref{LastPage}} % 页脚内容

\renewcommand{\headrulewidth}{0mm} % 页眉线宽

\title{SKA203 Assignment 1 \\ {\footnotesize adsadsa sadas dsadasdas}}
%\title{A Squid-like Bionic Robot} % 标题
%\author{Yuliang Xiao} % 作者
\author{\IEEEauthorblockN{San Zhang}
\IEEEauthorblockA{\textit{School of advanced technology} \\
\textit{Xi'an Jiaotong-liverpool University}\\
Suzhou, China \\
San.ZHang@student.xjtlu.edu.cn}} % 作者
%\date{\today}
\date{} % 日期

\lstset{
%    alsolanguage = Java,
	alsolanguage = [ANSI]C,
	alsolanguage = matlab,
%    alsolanguage = Python,
	basicstyle = \normalsize\setmainfont{JetBrains Mono}, %设置字体族
	breaklines = true, %自动换行
	keywordstyle = \bfseries\color{NavyBlue}, %设置关键字为粗体,颜色为 NavyBlue
	morekeywords = {}, %设置更多的关键字,用逗号分隔
	emph = {self}, %指定强调词,如果有多个,用逗号隔开
    emphstyle = \bfseries\color{Rhodamine}, %强调词样式设置
    commentstyle = \color{black!50!white}, %设置注释样式,浅灰色
    stringstyle = \bfseries\color{PineGreen!90!black}, %设置字符串样式
    columns = flexible, %使代码更紧凑
    %numbers = left, %显示行号在左边
    %numbersep = 0em, %设置行号的具体位置
    %numberstyle = \footnotesize, % 缩小行号
    frame = single, %边框
    framesep = 0.5 em, %设置代码与边框的距离
}